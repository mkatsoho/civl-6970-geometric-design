\documentclass{article}
% \documentclass{beamer}
% \usetheme{Madrid}

%------------------------------
% used for href/url
\usepackage{hyperref}


% used for math fomulas
\usepackage{amsmath}

% used for pictures
\usepackage{graphicx}
\usepackage{subcaption}
\usepackage{float}

% used for codes
\usepackage{listings}
\lstset{
  basicstyle=\ttfamily\footnotesize, % Set your code to be drawn with a monospaced font
  breaklines=true, % Enables line breaking
  frame=single % Adds a frame around the code
}

% used for Graph
\usepackage{adigraph}

% used for Bayesian Network
\usepackage{tikz}
\usetikzlibrary{bayesnet}

%------------------------------
% TODO 
\tolerance=10000
\emergencystretch=\maxdimen
\hyphenpenalty=10000
\hbadness=10000


\usepackage{silence}
\WarningFilter{latex}{Overfull \hbox}

%------------------------------



\title{MK's Notes for CIVL-4530 Geometric Design}
\date{2024-03-04}
\author{Michael Chen}

\begin{document}
  \pagenumbering{gobble}
  % \maketitle
  % \newpage
  \pagenumbering{arabic}
  % \setcounter{page}{29}

  % todo ----------------------------------------
  % \tableofcontents
  % \newpage

  % ----------------------------------------
  \section{Sight Distance (SD)}
  \subsection{Objectives}
  \begin{enumerate}
    \item describe various types of sight distance
    \item determine sight distance requirements for stopping and passing maneuvers
  \end{enumerate}

  \subsection{key component of SD}
  \begin{enumerate}
    \item PRT: the perception-reaction time required to initiate a maneuver (pre-maneuver phase)
    \item MT: the time requried to safely complete a maneuver
  \end{enumerate}
  driver's eye - 3.5ft high\\
  Hazard - 2ft high \\


  \subsection{Sight Distance Types}
  \begin{enumerate}
    \item stopping sight distance (SSD)
    \item decision sight distance (DSD)
    \item passing sight distance (PSD)
    \item intersection sight distance (ISD)
  \end{enumerate}

  \subsection{SSD - stopping sight distance}
  SSD is a key input for geometric design, including horizontal and vertical alignment \\
  \\
  PRT includes: recognize an object + decide a stop + react and prepare to apply the brake \\
  Deceleration rate: $11.2ft/sec^{2}$, 10th percentile deceleration rate, by AASHTO \\
  \begin{align*}
    SSD & = D_{p-r} + D_{b}\\
        & \textbf{$D_{p-r}$: in ft, perception-reaction distance} \\
        & \textbf{$D_{b}$: in ft, braking distance} \\
        \\
    D_{p-r} & = 1.47 \times 2.5s \times v = 3.675v \\
            & \textbf{$D_{p-r}$: in ft, perception-reaction distance} \\
            & \textbf{v: in mi/h, design speed} \\
            \\
    D_{b} & = \frac{(v_{0})^2 - (v_{f})^2}{30(\frac{a}{g} \pm G)} \\
          & \textbf{$D_{b}$: in ft, braking distance}\\
          & \textbf{$v_{0}$: in mi/h, design speed} \\
          & \textbf{$v_{f}$: in mi/h, final velocity}\\
          & \textbf{a: 11.2 $ft/sec^2$, deceleration rate, by AASHTO, in [10, 15]} \\
          & \textbf{g: 32.2 $ft/sec^2$} \\
          & \textbf{f = a/g: 0.35 by ASSHTO, coefficient of friction, 0.7 for dry roads, 0.3-0.4 for wed roads} \\
          & \textbf{G: grade, e.g. down grade: -0.06} \\
  \end{align*}

  \subsection{SSD on vertical curve}
  crest curve: \\
    - Driver eye height: 3.5ft \\
    - Height of object in readway: 2.0ft \\
    \\
  sag curve: \\
    - headlight height: 2ft \\
    - headlight beam angle: 1 degree (departure from horizontal, suggest changing to 0.75 degree)


  \subsection{DSS - decision sight distance}



  % ----------------------------------------
  \subsection{Terms}

  \subsection{Rules}

  \subsection{Formulas}

  \subsection{Reference}


% --------------------------------------
\end{document}