\documentclass{article}
% \documentclass{beamer}
% \usetheme{Madrid}

%------------------------------
% used for href/url
\usepackage{hyperref}


% used for math fomulas
\usepackage{amsmath}

% used for tables
\usepackage{booktabs} % For better looking tables

% used for pictures
\usepackage{graphicx}
\usepackage{subcaption}
\usepackage{float}

% used for codes
\usepackage{listings}
\lstset{
  basicstyle=\ttfamily\footnotesize, % Set your code to be drawn with a monospaced font
  breaklines=true, % Enables line breaking
  frame=single % Adds a frame around the code
}

% used for Graph
\usepackage{adigraph}

% used for Bayesian Network
\usepackage{tikz}
\usetikzlibrary{bayesnet}

%------------------------------
% TODO 
\tolerance=10000
\emergencystretch=\maxdimen
\hyphenpenalty=10000
\hbadness=10000


\usepackage{silence}
\WarningFilter{latex}{Overfull \hbox}

%------------------------------



\title{MK's Notes for CIVL-4530 Geometric Design}
\date{2024-03-04}
\author{Michael Chen}

\begin{document}
  \pagenumbering{gobble}
  % \maketitle      % hide title page
  % \newpage
  \pagenumbering{arabic}
  \setcounter{page}{30}   % start from page 30
  % \pagenumbering{roman}


  % ----------------------------------------
  \tableofcontents
  \newpage


  % ----------------------------------------
  % Geometric Design is a basic course of Transportation, introducing principal concepts and fomulas.
  % \newpage

  % ----------------------------------------
  \setcounter{section}{2}
  \section{Chapter 03 Sight Distance (SD)}
  \subsection{Objectives}
  \begin{enumerate}
    \item describe various types of sight distance
    \item determine sight distance requirements for stopping and passing maneuvers
  \end{enumerate}

  \subsection{key component of SD}
  \begin{enumerate}
    \item PRT: the perception-reaction time required to initiate a maneuver (pre-maneuver phase)
    \item MT: the time requried to safely complete a maneuver
  \end{enumerate}
  driver's eye - 3.5ft high\\
  Hazard - 2ft high \\


  \subsection{Sight Distance Types}
  \begin{enumerate}
    \item stopping sight distance (SSD)
    \item decision sight distance (DSD)
    \item passing sight distance (PSD)
    \item intersection sight distance (ISD)
  \end{enumerate}

  \subsection{SSD - stopping sight distance}
  SSD is a key input for geometric design, including horizontal and vertical alignment \\
  \\
  PRT includes: recognize an object + decide a stop + react and prepare to apply the brake \\
  Deceleration rate: $11.2ft/sec^{2}$, 10th percentile deceleration rate, by AASHTO \\
  \begin{align*}
    SSD & = D_{p-r} + D_{b}\\
        & \textbf{$D_{p-r}$: in ft, perception-reaction distance} \\
        & \textbf{$D_{b}$: in ft, braking distance} \\
        \\
    D_{p-r} & = 1.47 \times 2.5s \times v = 3.675v \\
            & \textbf{$D_{p-r}$: in ft, perception-reaction distance} \\
            & \textbf{v: in mi/h, design speed} \\
            \\
    D_{b} & = \frac{(v_{0})^2 - (v_{f})^2}{30(\frac{a}{g} \pm G)} \\
          & \textbf{$D_{b}$: in ft, braking distance}\\
          & \textbf{$v_{0}$: in mi/h, design speed} \\
          & \textbf{$v_{f}$: in mi/h, final velocity}\\
          & \textbf{a: 11.2 $ft/sec^2$, deceleration rate, by AASHTO, in [10, 15]} \\
          & \textbf{g: 32.2 $ft/sec^2$} \\
          & \textbf{f = a/g: 0.35 by ASSHTO, coefficient of friction, 0.7 for dry roads, 0.3-0.4 for wed roads} \\
          & \textbf{G: grade, e.g. down grade: -0.06} \\
  \end{align*}

  \subsection{SSD on vertical curve}
  crest curve: \\
    - Driver eye height: 3.5ft \\
    - Height of object in readway: 2.0ft \\
    \\
  sag curve: \\
    - headlight height: 2ft \\
    - headlight beam angle: 1 degree (departure from horizontal, suggest changing to 0.75 degree)


  \subsection{DSS - decision sight distance}
  For A or B (avoidance maneuvers): \\
  $DSD = 1.47V_{t} + 1.075(V^2/a)$ \\
  \\
  For C, D, and E: \\
  $DSD = 1.47V_{t}$ \\
  \\

  \subsection{DSS - decision sight distance}
  Decision sight distance for various conditions: \\
  \\
  Avoidance Maneuver A: Stop on rural road, t = 3.0 s\\
  Avoidance Maneuver B: Stop on urban road, t = 9.1 s\\
  \\
  Avoidance Maneuver C: Speed/path/direction change on rural road, t varies between 10.2 and 11.2 s \\
  Avoidance Maneuver D: Speed/path/direction change on suburban road, t varies between 12.1 and 12.9 s \\
  Avoidance Maneuver E: Speed/path/direction change on urban road, t varies between 14.0 and 14.5 s \\
  \\
  Source: AASHTO Green Book, 2011, Table 3-3\\



  \begin{table}[h!]
  \centering
  \caption{U.S. Customary Decision Sight Distance}
  \label{tab:us_customary_decision_sight_distance}
  \begin{tabular}{ccccccc}
  \toprule
  Design Speed (mph) & \multicolumn{5}{c}{Decision Sight Distance (ft)} &  \\
  \cmidrule(lr){2-6}
  & A & B & C & D & E & \\
  \midrule
  30 & 220 & 490 & 450 & 535 & 620 & \\
  35 & 275 & 590 & 525 & 625 & 720 & \\
  40 & 330 & 690 & 600 & 715 & 825 & \\
  45 & 395 & 800 & 675 & 800 & 930 & \\
  50 & 465 & 910 & 750 & 890 & 1030 & \\
  55 & 535 & 1030 & 865 & 980 & 1135 & \\
  60 & 610 & 1150 & 990 & 1125 & 1280 & \\
  65 & 695 & 1275 & 1050 & 1220 & 1365 & \\
  70 & 780 & 1410 & 1105 & 1275 & 1445 & \\
  75 & 875 & 1545 & 1180 & 1365 & 1545 & \\
  80 & 970 & 1685 & 1260 & 1455 & 1650 & \\
  \bottomrule
  \end{tabular}
  \end{table}

  \begin{table}[h!]
  \centering
  \caption{Metric Decision Sight Distance}
  \label{tab:metric_decision_sight_distance}
  \begin{tabular}{ccccccc}
  \toprule
  Design Speed (km/h) & \multicolumn{5}{c}{DSD (m)} \\
  \cmidrule(lr){2-6}
  & A & B & C & D & E \\
  \midrule
  50 & 70 & 155 & 145 & 170 & 195 \\
  60 & 95 & 195 & 170 & 205 & 235 \\
  70 & 115 & 325 & 200 & 235 & 275 \\
  80 & 140 & 280 & 230 & 270 & 315 \\
  90 & 170 & 325 & 270 & 315 & 360 \\
  100 & 200 & 370 & 315 & 355 & 400 \\
  110 & 235 & 420 & 330 & 380 & 430 \\
  120 & 265 & 470 & 360 & 415 & 470 \\
  130 & 305 & 525 & 390 & 450 & 510 \\
  \bottomrule
  \end{tabular}
  \end{table}

  \newpage

  \subsection{PSD - Passing sight distance}
  passing vehicle speed - passed vehicle speed $>=$ 12 mi/h\\
  \\
  On two-lane rural highways\\
  overtaking and returning to lane\\
  before opposing vehicle reaches passing vehicle\\


  \subsection{Passing sight distance assumptions - Green Book}
  \begin{enumerate}
    \item Speeds of passing and opposing vehicles equal the design speed
    \item Speed differential between the passing and passed vehicle is 12 mi/h
    \item Design vehicle is passenger car for all vehicles involved
    \item Perception-reaction time to decide to abort is 1 second
    \item Deceleration rate in abort maneuver is $11.2 ft/sec^2$
    \item Headway at end of maneuver is 1 second
  \end{enumerate}



  \begin{table}
  \centering
  \caption{U.S. Customary Assumed Speeds and Passing Sight Distance}
  \label{tab:us_customary_speeds}
  \begin{tabular}{cccc}
  \toprule
  Design Speed (mph) & Passed Vehicle (mph) & Passing Vehicle (mph) & Passing Sight Distance (ft) \\
  \midrule
  20 & 8  & 20 & 400 \\
  25 & 13 & 25 & 450 \\
  30 & 18 & 30 & 500 \\
  35 & 23 & 35 & 550 \\
  40 & 28 & 40 & 600 \\
  45 & 33 & 45 & 700 \\
  50 & 38 & 50 & 800 \\
  55 & 43 & 55 & 900 \\
  60 & 48 & 60 & 1000 \\
  65 & 53 & 65 & 1100 \\
  70 & 58 & 70 & 1200 \\
  75 & 63 & 75 & 1300 \\
  80 & 68 & 80 & 1400 \\
  \bottomrule
  \end{tabular}
  \end{table}
  
  \begin{table}[h!]
  \centering
  \caption{Metric Passing Sight Distance}
  \label{tab:metric_passing_sight_distance}
  \begin{tabular}{cccc}
  \toprule
  Design Speed (km/h) & Assumed Speeds Passed Vehicle (km/h) & Passing Vehicle (km/h) & PSD (m) \\
  \midrule
  30 & 11 & 30 & 120 \\
  40 & 21 & 40 & 140 \\
  50 & 31 & 50 & 160 \\
  60 & 41 & 60 & 180 \\
  70 & 51 & 70 & 210 \\
  80 & 61 & 80 & 245 \\
  90 & 71 & 90 & 280 \\
  100 & 81 & 100 & 320 \\
  110 & 91 & 110 & 355 \\
  120 & 101 & 120 & 395 \\
  130 & 111 & 130 & 440 \\
  \bottomrule
  \end{tabular}
  \end{table}
  
  


  % todo ----------------------------------------



  \subsection{vertical sight distance - Sighting Rod and Target Rod - AASHTO}
  Sighting rod (driver eye, used by observer):  3.5ft tall \\
  \\
  Target rod (object, used by assistant): 4.25ft tall
  \\


  \subsection{ISD - Intersection Sight Distance}
  \begin{enumerate}
    \item \textbf{a.k.a Green Book/PGDHS:} A Policy on Geometric Design of Highways and Streets, 2018, 7th Edition
    \item Guidelines for Geometric Design of Very Low Volume Local Roads, 2001
    \item A Guide to Achieving Flexibility in Highway Design, May 2004
    \item Guide for the Planning, Design, and Operation of Pedestrian Facilities, July 2004
    \item Guide for the Development of Bicycle Facilities, June 2012

    \item Good for New Highway Design 
    \item TRB Special Report 214, Designing Safer Roads: Practices for Resurfacing, Restoration, and Rehabilitation for guidance. \\
  \end{enumerate}

  

  \subsection{ISD - formula}
  The Intersection Sight Distance (ISD) is given by the formula:
  \begin{equation}
      ISD = 1.47 V_{\text{major}} \times t_g
  \end{equation}
  where:
  \begin{itemize}
      \item $ISD$ in ft, the intersection sight distance (length of the leg of sight triangle along the major road)
      \item $V_{\text{major}}$ in mph, the design speed of the major road in miles per hour (mph).
      \item $t_g$ in seconds, is the time gap for a minor road vehicle to enter the major road
  \end{itemize}
  where $t_g$:
  \begin{itemize}
      \item design vehicle: passenger car 7.5s; single-unit truck 9.5s; combination truck 11.5s
      \item a stopped vehicle turns \textbf{left} to a 2-lane highway; without median; grade $<= 3\%$
      \item Major highway: each additional lane, passenger car $+ 0.5s$; trunck $+ 0.7s$;
      \item Minor road: grade $> 3\%$, add 0.3s for each percent grade
  \end{itemize}






  \subsection{design elements}
  Design elements affect design consistency, driver expectancy, and vehicular operation.
  \begin{enumerate}
    \item horizontal and vertical alignment
    \item embankments and slopes
    \item shoulders, crown and cross slope, superelevation
    \item bridge widths
    \item signing and delineation
    \item guardrail and placement of utility poles or light supports
  \end{enumerate}

  \subsection{Highway Design Control Factors}
  \begin{enumerate}
    \item Highway Function (Arterials, Collections, Locals)
    \item Design speed of the facility
    \item Physical characteristics of the "design vehicle" 
    \item Performance of the design vehicle (heavy trucks, RVs)
    \item Acceptable degree of congestion
  \end{enumerate}

  \subsection{Highway functions}
    Highway Function: Arterials, Collections, Locals \\
    Arterials: principal arterials, minor arterials
    Mobility: the ability to move goods and passengers to their destination in a reasonable time 
    Accessibility: the ability to reach desired destination

  \subsection{Hierarchy of Movements - 6 stages}
  Main Movement \\
  Transition \\
  Distribution \\
  Collection \\
  Access \\
  Termination \\

  \subsection{Hierarchy of Movements}
	\begin{tabular}{|l|p{2cm}|p{2cm}|p{2cm}|p{2cm}|p{2cm}|}
	\hline
	\textbf{Roadway Class} & \textbf{\% Through Movement} & \textbf{VMT in Rural} & \textbf{Miles in Rural} & \textbf{VMT in Urban} & \textbf{Miles in Urban} \\
	\hline
	Freeways      & 100\%     &     &  \\
	Arterials     & 60-80\%   & {\bfseries 45-75\%} & 6-12\%  & {\bfseries 65-80\%}   & 15-25\% \\
	Collectors    & 40-60\%   & 20-35\% & 20-25\% & 5-19\%    & 5-10\% \\
	Local Streets & 0-40\%    & 5-20\%  & {\bfseries 65-75\%} & 10-30\%   & {\bfseries 65-80\%} \\
	\hline
	\end{tabular}

  \subsection{Highway Design Volume}
  \begin{tabular}{|l|l|l|}
  \hline
  Highway Type & Approximate Design Speed & Approximate Design Volume \\
  \hline
  Freeway – free flow & 70-75 mph & 2400 veh/h/ln \\
  Freeway – free flow & 65 mph & 2300 veh/h/ln \\
  \hline
  Rural Highways & & \\
  a) Multilane-one way & & 1600-2000 veh/h/ln \\
  b) Two lane & & 2000-2800 veh/h \\
  \hline
  Urban Highways & & \\
  a) Arterials & & See Highway Capacity Manual \\
  b) Signalized intersections & & 1900 pc/h/ln \\
  c) Unsignalized intersections & & 1100-2000 veh/h \\
  \hline
  \end{tabular}

  \subsection{Traffic Information for Roadway Designers}
  These traffic information should be available to the designer prior to or very early in the design process:
  \begin{enumerate}
    \item AADT for the current year: opening year (completion of construction), and design year
    \item Existing hourly traffic volumes over a minimum of 24-hour period, including peak hour turning movements and pedestrian counts
    \item Directional distribution factor (D30).
    \item 30th highest hour factor (K30).
    \item Truck factors (T) for daily and peak hour.
    \item Design speed and proposed posted speed.
    \item Design vehicle for geometric design.
    \item Turning movements and diagrams for existing and proposed signalized intersections.
    \item Special or unique traffic conditions, including during construction.
    \item Crash history, including analyses at high crash locations within the project limits.
    \item Recommendations regarding parking or other traffic restrictions.
  \end{enumerate}




  % ----------------------------------------
  \subsection{Terms}
  \begin{enumerate}
    \item PRT - perception-reaction time
    \item MT - maneuver time
    \item trajectory -  
    \item 
  \end{enumerate}


  \subsection{Rules}

  % ----------------------------------------
  \subsection{Formulas}

  % ----------------------------------------

  \subsection{Reference}

  \newpage





% --------------------------------------
\end{document}