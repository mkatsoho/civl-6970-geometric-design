\documentclass{article}
% \documentclass{beamer}
% \usetheme{Madrid}

% to include other tex files, containing tex pictures
\usepackage{standalone}

%------------------------------
% used for href/url
\usepackage{hyperref}


% used for math fomulas
\usepackage{amsmath}

% used for pictures
\usepackage{graphicx}
\usepackage{subcaption}
\usepackage{float}
\usepackage{gensymb}  % for symbols like degree

% used for codes
\usepackage{listings}
\lstset{
  basicstyle=\ttfamily\footnotesize, % Set your code to be drawn with a monospaced font
  breaklines=true, % Enables line breaking
  frame=single % Adds a frame around the code
}

% used for Graph
\usepackage{adigraph}

% used for Bayesian Network
\usepackage{tikz}
\usetikzlibrary{bayesnet}

%------------------------------
% TODO 
\tolerance=10000
\emergencystretch=\maxdimen
\hyphenpenalty=10000
\hbadness=10000


\usepackage{silence}
\WarningFilter{latex}{Overfull \hbox}

%------------------------------

\title{MK's Notes for CIVL-4530 Geometric Design}
\date{2024-03-04}
\author{Michael Chen}

\begin{document}
  \pagenumbering{gobble}
  % \maketitle
  % \newpage
  \pagenumbering{arabic}
  % \setcounter{page}{40}


  % todo ----------------------------------------
  % \tableofcontents
  % \newpage


  % ----------------------------------------
  \section{Horizontal Alignment}
  \subsection{Objectives}
  \begin{enumerate}
    \item Horizontal Curve Elements
    \item Superelevation
    \item Design Horizontal Curve
  \end{enumerate}

  % include horizontal curve diagram 
  \subsection{Horizontal Curve Diagram}
  \documentclass[10pt]{article}
\usepackage{pgfplots}
\pgfplotsset{compat=1.15}
\usepackage{mathrsfs}
\usetikzlibrary{arrows}
\pagestyle{empty}
\begin{document}
\definecolor{xdxdff}{rgb}{0.49019607843137253,0.49019607843137253,1}
\definecolor{zzttqq}{rgb}{0.6,0.2,0}
\definecolor{qqwuqq}{rgb}{0,0.39215686274509803,0}
\definecolor{zzttff}{rgb}{0.6,0.2,1}
\definecolor{ffxfqq}{rgb}{1,0.4980392156862745,0}
\definecolor{ududff}{rgb}{0.30196078431372547,0.30196078431372547,1}
\definecolor{ttttff}{rgb}{0.2,0.2,1}
\begin{tikzpicture}[line cap=round,line join=round,>=triangle 45,x=1cm,y=1cm]
\clip(-12.64051402144941,-3.9593507585936787) rectangle (4.808963853490404,8.010534452526674);
\draw [shift={(-4.21,-3.59)},line width=2pt,color=qqwuqq,fill=qqwuqq,fill opacity=0.10000000149011612] (0,0) -- (56.96449287513786:0.41579375396997176) arc (56.96449287513786:123.15173376158558:0.41579375396997176) -- cycle;
\draw [shift={(-4.219993389672106,6.262799169584461)},line width=2pt,color=qqwuqq,fill=qqwuqq,fill opacity=0.10000000149011612] (0,0) -- (-33.035507124862136:0.41579375396997176) arc (-33.035507124862136:33.235143591720956:0.41579375396997176) -- cycle;
\draw [shift={(-4.21,-3.59)},line width=2pt,color=ffxfqq]  plot[domain=0.9942179574000067:2.1494032336791298,variable=\t]({1*8.254477572808591*cos(\t r)+0*8.254477572808591*sin(\t r)},{0*8.254477572808591*cos(\t r)+1*8.254477572808591*sin(\t r)});
\draw [line width=2pt] (-4.21,-3.59)-- (0.29,3.33);
\draw [line width=2pt] (-8.73,3.33)-- (-4.21,-3.59);
\draw [line width=2pt,color=zzttff] (-8.73,3.33)-- (0.29,3.33);
\draw [line width=2pt] (-4.2136412260419975,-3.9593507585936787) -- (-4.2136412260419975,8.010534452526674);
\draw [line width=2pt,domain=-12.64051402144941:4.808963853490404] plot(\x,{(-24.3486--4.5*\x)/-6.92});
\draw [line width=2pt,domain=-12.64051402144941:4.808963853490404] plot(\x,{(--62.5032--4.52*\x)/6.92});
\draw [line width=2pt,color=zzttff] (-4.218372257170212,4.664473326948841)-- (-4.219993389672106,6.262799169584461);
\draw [line width=2pt,color=qqwuqq] (-4.218372257170212,4.664473326948841)-- (-4.217018742119951,3.33);
\draw [line width=2pt,color=zzttqq] (-4.219993389672106,6.262799169584461)-- (0.29,3.33);
\begin{scriptsize}
\draw [fill=ttttff] (-4.21,-3.59) circle (2.5pt);
\draw[color=ttttff] (-3.9781441470749996,-3.554302134180038) node {$O$};
\draw [fill=ududff] (0.29,3.33) circle (2.5pt);
\draw[color=ududff] (0.4847088122026978,3.6667371355390075) node {$PT$};
\draw [fill=ududff] (-8.73,3.33) circle (2.5pt);
\draw[color=ududff] (-9.009248570111659,3.7226059113201995) node {PC};
\draw[color=ffxfqq] (-4.6572739452259535,4.937751784561122) node {$L$};
\draw[color=black] (-1.7189980838381527,-0.048536453910249716) node {$R$};
\draw[color=zzttff] (-4.671133737024952,3.6807043294843056) node {$LC$};
\draw [fill=ttttff] (-4.218372257170212,4.664473326948841) circle (2pt);
\draw[color=ttttff] (-3.742527686492015,4.9656861724517185) node {$PCurve$};
\draw [fill=ttttff] (-4.217018742119951,3.33) circle (2pt);
\draw[color=ttttff] (-3.7009483110950185,3.191852541398877) node {$PChord$};
\draw [fill=ttttff] (-4.219993389672106,6.262799169584461) circle (2pt);
\draw[color=ttttff] (-4.005863730672997,6.683651027723368) node {$PI$};
\draw[color=zzttff] (-4.047443106069995,5.664045869716617) node {$E$};
\draw[color=qqwuqq] (-4.061302897868994,4.239392087296224) node {$M$};
\draw[color=zzttqq] (-1.76057745923515,5.0774237240141025) node {$T$};
\draw[color=qqwuqq] (-3.82568643728601,-2.85594243691514) node {$\Delta$};
\draw [fill=xdxdff] (2.054288598301512,10.374072899468619) circle (2.5pt);
\draw[color=qqwuqq] (-3.188136014532053,6.4182743427627065) node {$\Delta$};
\end{scriptsize}
\end{tikzpicture}
\end{document}

  \subsection{Horizontal Curve diagram}
  \textbf{Terms:}
  \begin{enumerate}
    \item PI: point of intersection, intersection of tangent and curve
    \item PC: point of curve(where curve starts), intersection of tangent and curve
    \item PT: point of tangency(where curve ends), intersection of 2 tangents
    \item PCurve: intersection of line PI-O and curve PC-PT
    \item PChord: intersection of chord PC-PT and line PI-O
    \item 
    \item R: the radius of the curve
    \item D: in degree, degree of curve, the degree matches 100ft arc
    \begin{enumerate}
      \item highways:(small) D $->$ arc length of 100ft
      \item railroads(big) D $->$ cord length of 100ft
    \end{enumerate}

    \item $\Delta$ - the degree of the curve, \textbf{the same as the bearings of 2 tangents}
    \item L: in feet, curve length 
    \item T: line segment from PI to PT/PC
    \item E: external distance, line segment of PI-PCurve
    \item M: middle ordinate, line segment of PCurve-PChord 
    \item LC: chord PC-PT 
    \item curvature - how big is a curve, $curvature = 1/R$
  \end{enumerate}

  \subsection{Horizontal Curve formulas}
  \begin{align*}
     \frac{D}{100} & = \frac{360}{2\pi R} \\
     D & = \frac{360 \cdot 100}{2\pi R} \\
     L & = 100 \cdot \Delta / D \\
     L_{in\_meter} & = 30.48 \cdot \Delta / D \\
     T & = R \cdot tan \frac{1}{2}\Delta \\
     M & = R(1-cos \frac{1}{2} \Delta) \\
     E & = R(\frac{1}{cos \frac{1}{2} \Delta} - 1) \\
     LC & = 2R \cdot sin \frac{1}{2} \Delta \\
  \end{align*}


  \subsection{stationing calculations - for PI, PC and PT}
  Stationing is the concept of assigning distances along a line, such as a survey baseline (initial field survey) or center line (design) \\
  \\
  52+48.63 means 52 hundreds 48 ft and .63 ft\\
  \\
  Given PI, calculate the locations:
  \begin{enumerate}
    \item why???
    \item PC = PI - T
    \item PT = PC + L
  \end{enumerate}


  \subsection{bearings calculations}
  A bearing refers to the direction and orientation of a line\\
  \\
  N 73\degree 30'38''E
  \\
  52+48.63 means 52 hundreds 48 ft and .63 ft


  \subsection{types of horizontal curves}
  \begin{enumerate}
    \item simple 
    \item compound (R1 and R2)
    \item reverse 
    \item spiral (change radius btw 2 Rs)
  \end{enumerate}
  VDOT: The use of spiral transitions for compound and reverse curves on urban roadways should be avoided, ...


  \subsection{sight distance on curves}
  terms:
  \begin{enumerate}
    \item highway centerline $->$ highway radius 
    \item centerline inside a lane $-> R_v $
    \item line of sight
    \item sight obstruction
    \item sight distance
  \end{enumerate}
    sight distance fomula:
  \begin{align*}
    & M_s = R_v \cdot (1 - cos_{degrees} \frac{90 SSD}{\pi R_v}) \\
    & M_s = R_v \cdot (1 - cos_{radians} \frac{SSD}{R_v}) \\
    \\
    & M_s: \textbf{the middle ordinate, distance from centerline to obstruction} \\
    & R_v: \textbf{the radius of the curve, inside lance | for the centerline of the *1st* lane}\\
    & SSD: \textbf{the stopping sight distance}
  \end{align*}

  \subsection{superelevation - centrifugal force}
  \begin{align*}
    & \textbf{suggestion value:} \\
    & 1 mph = 1.47 ft/sec \\
    \\
    & \textbf{superelevation formula:} \\
    & \frac{e + f}{1 - ef} = \frac{v^2}{gR} = \frac{V^2}{15R} \\
    \\
    & \textbf{minimam radius:} \\
    & R_{min} = \frac{V^2}{15(f_s + e_{max})} = \frac{v^2}{g(f_s + e_{max})} \\
    \\
    & \text{ e: superelevation rate, e.g. 0.04} \\
    & \text{ f: coefficient of friction} \\
    & \text{ $f_s$: side friction, e.g. 0.10} \\
    & \text{ g: 9.8, the acceleration of gravity} \\
    & \text{ v: in ft/sec, the speed in ft/sec} \\
    & \text{ V: in mph, the speed in mph} \\
    & \text{ R: in ft, the radius of the curve in feet} \\
  \end{align*}

  \subsection{superelevation - selection of e}
  \begin{enumerate}
    \item too high or too low?
    \item What factors should be considered?
    \item 
    \item e $\leq$ 0.10 on any paved road
    \item e $\leq$ 0.12 on unpaved roads
    \item e $\leq$ 0.08 where there is ice and snow
    \item e $\leq$ 0.06 in Illinois where ever practical
    \item e $\leq$ 0.04 in Illinois for urban freeways
  \end{enumerate}




  % todo ----------------------------------------

  \subsection{superelevation - transition design control}
  \begin{enumerate}
    \item Tangent runout (TR): the distance needed to change from a normal crown section to a point where the outside lane(s) is level
    \item Superelevation Runoff (L): the distance needed to change the cross slope from the end of TR to the design full superelevation rate. 
    \item IDOT practice: TR and 1/3 runoff on the tengent; 2/3 of runoff on the curve;
    \item ASSHTO: placing the PC at between 60\% and 80\% of the transition length
    \item ALDOT: 80/20 split (of the entire STL) 
    \item VDOT: this split is of the superelevation runoff portion only, not the entire STL
    \item Many states, including Virginia, use 2:1 split
  \end{enumerate}

  \subsection{axis of rotation}
  Axis of Rotation is the point about which the pavement edges are revolved to superelevate the roadway. 
  \begin{enumerate}
    \item typically, on undivided highways the centerline of roadway is the axis of rotation.  
    \item typicallly, divided highways rotate around the median edge
    \item also, some roadway revolved about inside edge!
  \end{enumerate}

  \subsection{superelevation runoff fomula}
  \begin{align*}
    & \textbf{superelevation runoff for *TWO-Lane* roads:} \\
    & L_{sro} \cong 30e(V + 32) \text{, for 12 ft lans} \\
    & L_{sro} \cong 25e(V + 32) \text{, for 10 ft lans} \\
    & L_{sro}: \text{superelevation runoff length} \\
    & \text{e: superelevation rate} \\
    \\
    & \textbf{full superelevation curve length:}\\
    & L_{full} = L_{curve} - 2L_{sro} \\
    & L_{full}: \text{full superelevation curve length} \\
    & L_{curve}: \text{horizontal curve length (arc length)} \\
    \\
    & \textbf{for multilane roads: (AASHTO Green Book) } \\
    & \text{times 1.5 for 4 lanes (2 in each direction)} \\
    & \text{times 2.0 for 6 lanes } \\
  \end{align*}

  \subsection{tangent runout formula}
  \begin{align*}
  & \textbf{tangent runout formula, given runoff:} \\
  & TR = L_{sro} \frac{NC}{e} \\
  & \text{TR: tengent runout} \\
  & \text{$L_{sro}$: superelevation runoff length} \\
  & \text{NC: normal crown rate} \\
  \end{align*}


  % ----------------------------------------
  \subsection{Terms}
  gradient: slope rate

  \subsection{Rules}

  \subsection{Formulas}

  \subsection{Reference}


% --------------------------------------
\end{document}