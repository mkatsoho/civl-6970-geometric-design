\documentclass{article}
% \documentclass{beamer}
% \usetheme{Madrid}

%------------------------------
% used for href/url
\usepackage{hyperref}


% used for math fomulas
\usepackage{amsmath}

% used for pictures
\usepackage{graphicx}
\usepackage{subcaption}
\usepackage{float}
\usepackage{gensymb}  % for symbols like degree

% used for codes
\usepackage{listings}
\lstset{
  basicstyle=\ttfamily\footnotesize, % Set your code to be drawn with a monospaced font
  breaklines=true, % Enables line breaking
  frame=single % Adds a frame around the code
}

% used for Graph
\usepackage{adigraph}

% used for Bayesian Network
\usepackage{tikz}
\usetikzlibrary{bayesnet}

%------------------------------
% TODO 
\tolerance=10000
\emergencystretch=\maxdimen
\hyphenpenalty=10000
\hbadness=10000


\usepackage{silence}
\WarningFilter{latex}{Overfull \hbox}

%------------------------------



\title{MK's Notes for CIVL-4530 Geometric Design}
\date{2024-03-04}
\author{Michael Chen}

\begin{document}
  \pagenumbering{gobble}
  % \maketitle
  % \newpage
  \pagenumbering{arabic}
  \setcounter{page}{40}


  % ----------------------------------------
  \tableofcontents
  \newpage


  % ----------------------------------------
  \setcounter{section}{3}
  \section{Chapter 04 horizontal Alignment}
  \subsection{Objectives}
  \begin{enumerate}
    \item Horizontal Curve Elements
    \item Superelevation
    \item Design Horizontal Curve
  \end{enumerate}

  

  \subsection{Horizontal Curve diagram}
  Terms:
  \begin{enumerate}
    \item PI: point of intersection, intersection of tangent and curve
    \item PC: point of curve(where curve starts), intersection of tangent and curve
    \item PT: point of tangency(where curve ends), intersection of 2 tangents
    \item PCurve: intersection of line PI-O and curve PC-PT
    \item PChord: intersection of chord PC-PT and line PI-O
    \item 
    \item R: the radius of the curve
    \item D: in degree, degree of curve, the degree matches 100ft arc
    \begin{enumerate}
      \item highways:(small) D $->$ arc length of 100ft
      \item railroads(big) D $->$ cord length of 100ft
    \end{enumerate}

    \item $\Delta$ - the degree of the curve, \textbf{the same as the bearings of 2 tangents}
    \item L: in feet, curve length 
    \item T: line segment from PI to PT/PC
    \item E: line segment of PI-PCurve
    \item M: middle ordinate, line segment of PCurve-PChord 
    \item LC: chord PC-PT 
    \item curvature - how big is a curve, $curvature = 1/R$
  \end{enumerate}
  Formulas:
  \begin{align*}
     \frac{D}{100} & = \frac{360}{2\pi R} \\
     D & = \frac{360 \cdot 100}{2\pi R} \\
     L & = 100 \cdot \Delta / D \\
     L_{in\_meter} & = 30.48 \cdot \Delta / D \\
     T & = R \cdot tan \frac{1}{2}\Delta \\
     M & = R(1-cos \frac{1}{2} \Delta) \\
     E & = R(\frac{1}{cos \frac{1}{2} \Delta} - 1) \\
     LC & = 2R \cdot sin \frac{1}{2} \Delta \\
  \end{align*}


  \subsection{stationing calculations}
  Stationing is the concept of assigning distances along a line, such as a survey baseline (initial field survey) or center line (design) \\
  \\
  52+48.63 means 52 hundreds 48 ft and .63 ft\\
  \\
  Given PI, calculate the locations:
  \begin{enumerate}
    \item why???
    \item PC = PI - T
    \item PT = PC + L
  \end{enumerate}


  \subsection{bearings calculations}
  A bearing refers to the direction and orientation of a line\\
  \\
  N 73\degree 30'38''E
  \\
  52+48.63 means 52 hundreds 48 ft and .63 ft
  \begin{enumerate}
    \item 
  \end{enumerate}


  \subsection{types of horizontal curves}
  \begin{enumerate}
    \item simple 
    \item compound (R1 and R2)
    \item reverse 
    \item spiral (change radius btw 2 Rs)
  \end{enumerate}
  VDOT: The use of spiral transitions for compound and reverse curves on urban roadways should be avoided, ...



  \subsection{sight distance on curves}
  terms:
  \begin{enumerate}
    \item highway centerline $->$ highway radius 
    \item centerline inside a lane $-> R_v $
    \item line of sight
    \item sight obstruction
    \item sight distance
  \end{enumerate}
    sight distance fomula:
  \begin{align*}
    & M_s = R_v \cdot (1 - cos_{degrees} \frac{90 SSD}{\pi R_v}) \\
    & M_s = R_v \cdot (1 - cos_{radians} \frac{SSD}{R_v}) \\
    \\
    & M_s: \textbf{the middle ordinate, distance from centerline to obstruction} \\
    & R_v: \textbf{the radius of the curve, inside lance | for the centerline of the *1st* lane}\\
    & SSD: \textbf{the stopping sight distance}
  \end{align*}

  \subsection{superelevation - centrifugal force}
  \begin{align*}
    & \textbf{suggestion value:} \\
    & 1 mph = 1.47 ft/sec \\
    \\
    & \textbf{superelevation formula:} \\
    & \frac{e + f}{1 - ef} = \frac{v^2}{gR} = \frac{V^2}{15R} \\
    \\
    & \textbf{minimam radius:} \\
    & R_{min} = \frac{V^2}{15(f_s + e_{max})} = \frac{v^2}{g(f_s + e_{max})} \\
    \\
    & \text{ e: the rate of superelevation} \\
    & \text{ f: coefficient of friction} \\
    & \text{ $f_s$: side friction} \\
    & \text{ g: 9.8, the acceleration of gravity} \\
    & \text{ v: in ft/sec, the speed in ft/sec} \\
    & \text{ V: in mph, the speed in mph} \\
    & \text{ R: in ft, the radius of the curve in feet} \\
  \end{align*}

  % todo ----------------------------------------
  \subsection{superelevation - selection of e}
  \begin{enumerate}
    \item too high or too low?
    \item What factors should be considered?
    \item 
    \item e $\leq$ 0.10 on any paved road
    \item e $\leq$ 0.12 on unpaved roads
    \item e $\leq$ 0.08 where there is ice and snow
    \item e $\leq$ 0.06 in Illinois where ever practical
    \item e $\leq$ 0.04 in Illinois for urban freeways
  \end{enumerate}

  \subsection{Highway Design Control Factors}
  \begin{enumerate}
    \item Highway Function (Arterials, Collections, Locals)
    \item Design speed of the facility
    \item Physical characteristics of the "design vehicle" 
    \item Performance of the design vehicle (heavy trucks, RVs)
    \item Acceptable degree of congestion
  \end{enumerate}

  \subsection{Highway functions}
    Highway Function: Arterials, Collections, Locals \\
    Arterials: principal arterials, minor arterials
    Mobility: the ability to move goods and passengers to their destination in a reasonable time 
    Accessibility: the ability to reach desired destination

  \subsection{Hierarchy of Movements - 6 stages}
  Main Movement \\
  Transition \\
  Distribution \\
  Collection \\
  Access \\
  Termination \\

  \subsection{Hierarchy of Movements}
	\begin{tabular}{|l|p{2cm}|p{2cm}|p{2cm}|p{2cm}|p{2cm}|}
	\hline
	\textbf{Roadway Class} & \textbf{\% Through Movement} & \textbf{VMT in Rural} & \textbf{Miles in Rural} & \textbf{VMT in Urban} & \textbf{Miles in Urban} \\
	\hline
	Freeways      & 100\%     &     &  \\
	Arterials     & 60-80\%   & {\bfseries 45-75\%} & 6-12\%  & {\bfseries 65-80\%}   & 15-25\% \\
	Collectors    & 40-60\%   & 20-35\% & 20-25\% & 5-19\%    & 5-10\% \\
	Local Streets & 0-40\%    & 5-20\%  & {\bfseries 65-75\%} & 10-30\%   & {\bfseries 65-80\%} \\
	\hline
	\end{tabular}

  \subsection{Highway Design Volume}
  \begin{tabular}{|l|l|l|}
  \hline
  Highway Type & Approximate Design Speed & Approximate Design Volume \\
  \hline
  Freeway – free flow & 70-75 mph & 2400 veh/h/ln \\
  Freeway – free flow & 65 mph & 2300 veh/h/ln \\
  \hline
  Rural Highways & & \\
  a) Multilane-one way & & 1600-2000 veh/h/ln \\
  b) Two lane & & 2000-2800 veh/h \\
  \hline
  Urban Highways & & \\
  a) Arterials & & See Highway Capacity Manual \\
  b) Signalized intersections & & 1900 pc/h/ln \\
  c) Unsignalized intersections & & 1100-2000 veh/h \\
  \hline
  \end{tabular}

  \subsection{Traffic Information for Roadway Designers}
  These traffic information should be available to the designer prior to or very early in the design process:
  \begin{enumerate}
    \item AADT for the current year: opening year (completion of construction), and design year
    \item Existing hourly traffic volumes over a minimum of 24-hour period, including peak hour turning movements and pedestrian counts
    \item Directional distribution factor (D30).
    \item 30th highest hour factor (K30).
    \item Truck factors (T) for daily and peak hour.
    \item Design speed and proposed posted speed.
    \item Design vehicle for geometric design.
    \item Turning movements and diagrams for existing and proposed signalized intersections.
    \item Special or unique traffic conditions, including during construction.
    \item Crash history, including analyses at high crash locations within the project limits.
    \item Recommendations regarding parking or other traffic restrictions.
  \end{enumerate}




  % ----------------------------------------
  \subsection{Terms}
  \begin{enumerate}
    \item cross section - A cross section refers to the vertical view of a roadway or highway at right angles to its centerline. 
    \item embankment - An embankment is a constructed mound of earth, stones, or other materials. Its purpose is to support the raising of a roadway or railway above the level of the surrounding ground surface.
    \item cross slope - Cross slope plays a crucial role in ensuring proper drainage and safety on roadways.
    \item crown - The crown of a highway refers to the cross-sectional shape of the road surface.
    \item signing and delineation -  
    \item guardrail - A guardrail on a highway serves as a safety barrier designed to protect motorists.
    \item guardrail and placement of utility poles or light supports - 
    \item  detour - walkaround roadway  
    \item through movement - refers to the uninterrupted flow of vehicles or goods from one location to another 
    \item VMT - Vehicle Miles Traveled
    \item open year and design year - open year means compeletion of construction. 
    \item $D_{30}$ factor - Directional Distribution factor 
    \item $K_{30}$ factor - the 30th highest hour factor 
  \end{enumerate}


  \subsection{Rules}

  % ----------------------------------------
  \subsection{Formulas}

  % ----------------------------------------

  \subsection{Reference}

  \newpage





% --------------------------------------
\end{document}