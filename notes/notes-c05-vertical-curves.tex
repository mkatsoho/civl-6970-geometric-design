\documentclass{article}
% \documentclass{beamer}
% \usetheme{Madrid}

% to include other tex files, containing tex pictures
\usepackage{standalone}

%------------------------------
% used for href/url
\usepackage{hyperref}


% used for math fomulas
\usepackage{amsmath}

% used for pictures
\usepackage{graphicx}
\usepackage{subcaption}
\usepackage{float}
\usepackage{gensymb}  % for symbols like degree

% used for codes
\usepackage{listings}
\lstset{
  basicstyle=\ttfamily\footnotesize, % Set your code to be drawn with a monospaced font
  breaklines=true, % Enables line breaking
  frame=single % Adds a frame around the code
}

% used for Graph
\usepackage{adigraph}

% used for Bayesian Network
\usepackage{tikz}
\usetikzlibrary{bayesnet}

%------------------------------
% TODO 
\tolerance=10000
\emergencystretch=\maxdimen
\hyphenpenalty=10000
\hbadness=10000


\usepackage{silence}
\WarningFilter{latex}{Overfull \hbox}

%------------------------------

\title{MK's Notes for CIVL-4530 Geometric Design}
\date{2024-03-10}
\author{Michael Chen}

\begin{document}
  % \pagenumbering{gobble}
  % \maketitle
  % \newpage
  % \pagenumbering{arabic}
  % \setcounter{page}{40}
  \setcounter{section}{4}

  % todo ----------------------------------------
  % \tableofcontents
  % \newpage


  % ----------------------------------------
  \section{Vertical Alignment}
  \subsection{Objectives}
  \begin{enumerate}
    \item Understand basic philosophies in establishing a vertical alignment 
    \item Apply criteria for selection of grades
    \item Design a vertical Curve
  \end{enumerate}

  \subsection{philosophies}
  \begin{enumerate}
    \item conform to the existing terrain (within constraints of max grade and min lengths of vertical curves)
    \item to minimize impacts (balance earthwork)
    \item coordinate horizontal and vertical alignments (HC and VC)
    \begin{enumerate}
      \item avoid steep (near the max) grades and sharp (near min radius) horizontal curve
      \item avoid placing the start of HC in the middle of VC; VC either at HC tangents or at HC vurves 
      \item avoid placing the start of HC at the bottom of a steep VC
    \end{enumerate}
  \end{enumerate}

  \subsection{maximum grades}
  \begin{enumerate}
    \item Steepness and length heavily impacts heavy vehicles.
    \item max grade design criteria is related with: design speed, the functional classification, and terrain
    \item max grades: by AASHTO
    \begin{enumerate}
      \item freeway: 3-6\%, +3.00\% 70mph; max +4.00\% for upgrade, max -5.00\% for downgrade 
      \item arterials: +3.00\% 60mph; up to +8.00\% 40mph at mountainous 
      \item collectors: +4.00\% 70mph; up to +14.00\% 20 mph, mountainous
      \item locals: up to +17.00\% in mountainous terrain
    \end{enumerate}
    \item min grades:
    \begin{enumerate}
      \item urban design(curb and gutter): an appropriate min grade is 0.5\%, but grade of .30\% ...
      \item rural deisgn(shoulder and ditches): ... cross-slope is adaquate ...
    \end{enumerate}
  \end{enumerate}

  \subsection{Vertical Curve}
  \input{Vertical Curve2.txt}  % TODO - crop the drawing
  \begin{align*}
    & \textbf{If $G_1$ and $G_2$ are in slope, e.g. +0.02, x and L must in feet}\\
    \\
    & A = G_2 - G_1  \\
    & \text{A: total change in grade, if negative, the curve is below the tangent}\\
    & \text{$G_n$: grade, like -0.08} \\
    \\
    & Y_{offset} = x^2 \frac{A}{2L} = x^2 \frac{G_2 - G_1}{2L} \\
    & Y_{offset}: \text{vertical offset from a tangent to a parabola, maybe negative!!!} \\
    \\
    & Y_{tan} = Y_{vpc} + G_1x \\
    & Y_{tan}: \text{tangent elevation} \\
    \\
    & Y_{curve} = Y_{vpc} + G_1x + \frac{A}{2L}x^2 \\
    & Y_{curve}: \text{curve elevation} \\
    \\
    & Y_{tan} = Y_{offset} + Y_{curve} \\
    \\
    & x_{hi/lo} = L \frac{G_1}{G_1 - G_2} \\
    & x_{hi/lo}: \text{the highest/lowest point}\\
  \end{align*}


  \subsection{stopping/passing sight distance on crest curves}
  \begin{align*}
    & A = G_2 - G_1 \\
    \\
    & \textbf{*SSD/PSD* minimum length of crest curve:} \\
    & L = \frac{|A|S^2}{200(\sqrt{h_1} + \sqrt{h_2})^2}  \text{   when } S \leq L \\
    & L = 2S - \frac{200(\sqrt{h_1} + \sqrt{h_2})^2}{|A|}  \text{  when } S \geq L \\
    & \text{ stopping: $h_1 = 3.5ft$,  $h_2 = 2.0 ft$ } \\
    & \text{ passing: $h_1 = 3.5ft$,  $h_2 = 3.5ft$} \\
    \\
    & \textbf{*SSD* minimum length of crest curve:} \\
    & L_{SSD} = \frac{|A|S^2}{2158} \text{   when } S \leq L \\
    & L_{SSD} = 2S - \frac{2158}{|A|} \text{  when } S \geq L \\
    \\
    & \textbf{A: in unit \%, e.g. 3, the grade change of VC}\\
    & \text{L: the minimum length of the vertical curve ??? the arc length or the horizontal segment length???}\\
    & \text{S: stop sight distance, related with speed, reaction time, and coefficient of friction} \\
    & h_1: \text{3.5 feet, the driver eye height} \\
    & h_2: \textbf{2.0 feet, the object height} \\
  \end{align*}

  \subsection{sag sight distance}
  TODO diagram??? \\
  \begin{align*}
    & \textbf{minimum length of sag curve} \\
    & L = \frac{|A|S^2}{200(h + S \cdot tan \beta)}   \text{ when } S \leq L \\
    & L = 2S - \frac{200(h + S \cdot tan \beta)}{|A|} \text{ when } S \geq L \\
    \\
    & L = \frac{|A|S^2}{400 + 3.5S} \text{ when } S \leq L \\
    & L = 2S - \frac{400 + 3.5S}{|A|} \text{ when } S \geq L \\
    & \text{A: in unit \%, e.g. 3, the grade change of VC}\\
    & \text{L: the horizontal length of sag curve}\\
    & \text{S: sag sight distance} \\
    & \text{h: 2 feet, the headlamp height}\\
    & \beta: \text{1 degree, the headlamp beam angle}\\
  \end{align*}

  % TODO
  \subsection{Vertical curve design - AASHTO elevation table - $K^a$}
  \begin{align*}
    & \textbf{Another approach to determining curve length! }\\
    &  L = K \cdot A \\
    \\
    & \text{L: in feet, curve length, minimum length for a given design speed }\\
    & \text{A: in unit \%, change in grade  }\\
    & \text{K: rate of vertical curvature, K = required ft of curve length per 1\% net change in grade }\\
    & \text{K table for crest VC }\\
    & \text{K table for sag VC }
  \end{align*}
  TODO add tables???

  \subsection{vertical alignment - elevation}
  TODO % TODO
  \begin{enumerate}
    \item 
  \end{enumerate}
  \begin{align*}
    \\
  \end{align*}


  % ----------------------------------------
  \subsection{Terms}
  gradient: slope rate
  grade: e.g. +4.00\% a upward slope; -3.00\% a downward slope

  \subsection{Rules}

  \subsection{Formulas}

  \subsection{Reference}


% --------------------------------------
\end{document}
